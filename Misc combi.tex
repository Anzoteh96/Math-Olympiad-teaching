\documentclass[11pt,a4paper]{article}
\usepackage{amsmath, amssymb, fullpage, mathrsfs, bm, pgf, tikz}

\begin{document}

\title{Caplang (miscellaneous) combinatorics}
\author{Anzo Teh}
\date{22 August 2016}
\maketitle

\section{Invariant}
\begin{enumerate}
\item\textbf{IMO 2011, \#2.} Let $\mathcal{S}$ be a finite set of at least two points in the plane. Assume that no three points of $\mathcal S$ are collinear. A windmill is a process that starts with a line $\ell$ going through a single point $P \in \mathcal S$. The line rotates clockwise about the pivot $P$ until the first time that the line meets some other point belonging to $\mathcal S$. This point, $Q$, takes over as the new pivot, and the line now rotates clockwise about $Q$, until it next meets a point of $\mathcal S$. This process continues indefinitely.\\
Show that we can choose a point $P$ in $\mathcal S$ and a line $\ell$ going through $P$ such that the resulting windmill uses each point of $\mathcal S$ as a pivot infinitely many times.

\item\emph{IMO 2007, C3.} Find all positive integers $ n$ for which the numbers in the set $ S = \{1,2, \ldots,n \}$ can be colored red and blue, with the following condition being satisfied: The set $ S \times S \times S$ contains exactly $ 2007$ ordered triples $ \left(x, y, z\right)$ such that:

(i) the numbers $ x$, $ y$, $ z$ are of the same color,
and
(ii) the number $ x + y + z$ is divisible by $ n$.

\item\emph {Raffles MO 2014,\#5.} 2015 equally spaced points have been painted blue or red. What's the minimum number of monochromatic isoceles triangles?
\end{enumerate}

\section{Tilings and boards.}
\begin{enumerate}
\item\textbf {IMO 2016, \# 2.} Find all integers $n$ for which each cell of $n \times n$ table can be filled with one of the letters $I,M$ and $O$ in such a way that:\\
$\bullet$  in each row and each column, one third of the entries are $I$, one third are $M$ and one third are $O$; and\\
$\bullet$ in any diagonal, if the number of entries on the diagonal is a multiple of three, then one third of the entries are $I$, one third are $M$ and one third are $O$.

\item\emph {IMO 2014, C4.} Construct a tetromino by attaching two $2 \times 1$ dominoes along their longer sides such that the midpoint of the longer side of one domino is a corner of the other domino. This construction yields two kinds of tetrominoes with opposite orientations. Let us call them $S$- and $Z$-tetrominoes, respectively.
Assume that a lattice polygon $P$ can be tiled with $S$-tetrominoes. Prove that no matter how we tile $P$ using only $S$- and $Z$-tetrominoes, we always use an even number of $Z$-tetrominoes.

\item\emph{IMO 2011, C7.} On a square table of $2011$ by $2011$ cells we place a finite number of napkins that each cover a square of $52$ by $52$ cells. In each cell we write the number of napkins covering it, and we record the maximal number $k$ of cells that all contain the same nonzero number. Considering all possible napkin configurations, what is the largest value of $k$? (What's the highest possible frequency of the mode, exluding 0?)
\end{enumerate}

\section{Optimization/Max min consideration}
\begin{enumerate}
\item\textbf{IMO 2014, \#2.} Let $n \ge 2$ be an integer. Consider an $n \times n$ chessboard consisting of $n^2$ unit squares. A configuration of $n$ rooks on this board is peaceful if every row and every column contains exactly one rook. Find the greatest positive integer $k$ such that, for each peaceful configuration of $n$ rooks, there is a $k \times k$ square which does not contain a rook on any of its $k^2$ unit squares.

\item\textbf{IMO 2013, \#2.} A configuration of $4027$ points in the plane is called Colombian if it consists of $2013$ red points and $2014$ blue points, and no three of the points of the configuration are collinear. By drawing some lines, the plane is divided into several regions. An arrangement of lines is good for a Colombian configuration if the following two conditions are satisfied:

i) No line passes through any point of the configuration.

ii) No region contains points of both colors.

Find the least value of $k$ such that for any Colombian configuration of $4027$ points, there is a good arrangement of $k$ lines.

\item\emph{IMO 2013, C4.} Let $n$ be a positive integer, and let $A$ be a subset of $\{ 1,\cdots ,n\}$. An $A$-partition of $n$ into $k$ parts is a representation of n as a sum $n = a_1 + \cdots + a_k$, where the parts $a_1 , \cdots , a_k $ belong to $A$ and are not necessarily distinct. The number of different parts in such a partition is the number of (distinct) elements in the set $\{ a_1 , a_2 , \cdots , a_k \} $.
We say that an $A$-partition of $n$ into $k$ parts is optimal if there is no $A$-partition of $n$ into $r$ parts with $r<k$. Prove that any optimal $A$-partition of $n$ contains at most $\sqrt[3]{6n}$ different parts.

\item\emph{IMO 2008, C5.} Let $ S = \{x_1, x_2, \ldots, x_{k + l}\}$ be a $ (k + l)$-element set of real numbers contained in the interval $ [0, 1]$; $ k$ and $ l$ are positive integers. A $ k$-element subset $ A\subset S$ is called nice if
\[ \left |\frac {1}{k}\sum_{x_i\in A} x_i - \frac {1}{l}\sum_{x_j\in S\setminus A} x_j\right |\le \frac {k + l}{2kl}\]
Prove that the number of nice subsets is at least $ \dfrac{2}{k + l}\dbinom{k + l}{k}$.

\item\emph{APMO 2012, \# 2.} Into each box of a $ 2012 \times 2012 $ square grid, a real number greater than or equal to $ 0 $ and less than or equal to $ 1 $ is inserted. Consider splitting the grid into $2$ non-empty rectangles consisting of boxes of the grid by drawing a line parallel either to the horizontal or the vertical side of the grid. Suppose that for at least one of the resulting rectangles the sum of the numbers in the boxes within the rectangle is less than or equal to $ 1 $, no matter how the grid is split into $2$ such rectangles. Determine the maximum possible value for the sum of all the $ 2012 \times 2012 $ numbers inserted into the boxes.
\end{enumerate}

\section{Mappings}
\begin{enumerate}
\item\textbf {IMO 2008, \# 5.} Let $ n$ and $ k$ be positive integers with $ k \geq n$ and $ k - n$ an even number. Let $ 2n$ lamps labelled $ 1$, $ 2$, ..., $ 2n$ be given, each of which can be either on or off. Initially all the lamps are off. We consider sequences of steps: at each step one of the lamps is switched (from on to off or from off to on).

Let $ N$ be the number of such sequences consisting of $ k$ steps and resulting in the state where lamps $ 1$ through $ n$ are all on, and lamps $ n + 1$ through $ 2n$ are all off.

Let $ M$ be number of such sequences consisting of $ k$ steps, resulting in the state where lamps $ 1$ through $ n$ are all on, and lamps $ n + 1$ through $ 2n$ are all off, but where none of the lamps $ n + 1$ through $ 2n$ is ever switched on.

Determine $ \frac {N}{M}$.

\item\emph {IMO 2006, C3.} Let $ S$ be a finite set of points in the plane such that no three of them are on a line. For each convex polygon $ P$ whose vertices are in $ S$, let $ a(P)$ be the number of vertices of $ P$, and let $ b(P)$ be the number of points of $ S$ which are outside $ P$. A line segment, a point, and the empty set are considered as convex polygons of $ 2$, $ 1$, and $ 0$ vertices respectively. Prove that for every real number $ x$ \[\sum_{P}{x^{a(P)}(1 - x)^{b(P)}} = 1,\] where the sum is taken over all convex polygons with vertices in $ S$.
\end{enumerate}

\section{Induct!}
\begin{enumerate}
\item\textbf {IMO 2011,\# 4.} Let $n > 0$ be an integer. We are given a balance and $n$ weights of weight $2^0, 2^1, \cdots, 2^{n-1}$. We are to place each of the $n$ weights on the balance, one after another, in such a way that the right pan is never heavier than the left pan. At each step we choose one of the weights that has not yet been placed on the balance, and place it on either the left pan or the right pan, until all of the weights have been placed.\\
Determine the number of ways in which this can be done.

\item\emph {APMO 2008, \# 2.} Students in a class form groups each of which contains exactly three members such that any two distinct groups have at most one member in common. Prove that, when the class size is $ 46$, there is a set of $ 10$ students in which no group is properly contained.
\end{enumerate}


\end{document}