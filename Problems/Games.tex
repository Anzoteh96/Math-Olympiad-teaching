\documentclass[11pt,a4paper]{article}
\usepackage{amsmath, amssymb, fullpage, mathrsfs, bm, pgf, tikz}

\begin{document}

\title{Games}
\author{Anzo Teh}
\date{5 August 2016}
\maketitle
\section {TOT}
\begin{enumerate}
\item\emph {2015 FSO3.} Three players play the game "rock-paper-scissors".  In every round, each player
simultaneously  shows  one  of  these  shapes.   Rock  beats  scissors,  scissors  beat
paper,  while  paper  beats  rock.   If  in  a  round  exactly  two  distinct  shapes  are
shown (and thus one of them is shown twice) then 1 point is added to the score
of the player(s) who showed the winning shape,  otherwise no point is added.

After  several  rounds  it  occurred  that  each  shape  had  been  shown  the  same
number  of  times.   Prove  that  the  total  sum  of  points  at  this  moment  was  a
multiple of 3.

\item\emph {2015 SSA6.} An Emperor invited 2015 wizards to a festival.  Each of the wizards knows who of them is good
and who is evil, however the Emperor doesn't know this.  A good wizard always tells the truth,
while an evil wizard can tell the truth or lie at any moment.  The Emperor gives each wizard a
card with a single question, maybe di erent for di erent wizards, and after that listens to the
answers of all wizards which are either "yes" or "no".  Having listened to all the answers, the
Emperor expels a single wizard through a magic door which shows if this wizard is good or evil.
Then the Emperor makes new cards with questions and repeats the procedure with the remaining
wizards, and so on.  The Emperor may stop at any moment, and after this the Emperor may expel
or not expel a wizard.  Prove that the Emperor can expel all the evil wizards having expelled at
most one good wizard.

\item\emph {2014FSO3.} From a set of 15 distinct integers Pete selects 7 numbers in all possible ways
and for every selection he writes down the sum of the selected numbers.  Basil,
in his turn, selects 8 numbers in all possible ways and each time writes down
the sum of his selected numbers.  Can it happen that Pete and Basil will obtain
the same set of numbers?  (Each integer must be repeated in Pete's set as many
times as it is repeated in Basil's set.)

\item\emph {2014 SSA3.} The King called two wizards.  He ordered First Wizard to write down 100 positive real
numbers (not necessarily distinct) on cards without revealing them to Second Wizard.
Second  Wizard  must  correctly  determine  all  these  numbers,  otherwise  both  wizards
will lose their heads.  First Wizard is allowed to provide Second Wizard with a list of
distinct numbers, each of which is either one of the numbers on the cards or a sum of
some of these numbers.  He is not allowed to tell which numbers are on the cards and
which numbers are their sums.  If Second Wizard correctly determines all 100 numbers
the King tears as many hairs from each wizard's beard as the number of numbers in
the list given to Second Wizard.  What is the minimal number of hairs each wizard
should sacrifice to stay alive?

\item\emph {2014 SSA5.} There  is  a  scalene  triangle.   Peter  and  Basil  play  the  following  game.   On  each  his
turn  Peter  chooses  a  point  in  the  plane.   Basil  responds  by  painting  it  into  red  or
blue.  Peter wins if some triangle similar to the original one has all vertices of the same
colour.  Find the minimal number of moves Peter needs to win no matter how Basil
would play (independently of the shape of the given triangle)?

\item\emph {2012 FSA1.} There is an $8\times 8$ table, drawn in a plane and painted in a chess board fashion.
Peter mentally chooses a square and an interior point in it.  Basil can draws
any polygon (without self-intersections) in the plane and ask Peter whether the
chosen point is inside or outside this polygon.  What is the minimal number of
questions sufficient to determine whether the chosen point is black or white?

\item\emph {2013 FSA6.} On  a  table,  there  are  11  piles  of  ten  stones  each.   Pete  and  Basil  play  the following game.  In turns they take 1, 2 or 3 stones at a time:  Pete takes stones from any single pile while Basil takes stones from di erent piles but no more than one from each.  Pete moves  first.  The player who cannot move, loses. Which of the players, Pete or Basil, has a winning strategy?

\item\emph {2013 SSA7.} The King decided to reduce his Council consisting of thousand wizards.  He placed them in a
line and placed hats with numbers from 1 to 1001 on their heads not necessarily in this order (one
hat was hidden).  Each wizard can see the numbers on the hats of all those before him but not on
himself or on anyone who stayed behind him.  By King's command, starting from the end of the
line each wizard calls one integer from 1 to 1001 so that every wizard in the line can hear it.  No
number can be repeated twice.

In the end each wizard who fails to call the number on his hat is removed from the Council.  The
wizards knew the conditions of testing and could work out their strategy prior to it.

Can the wizards work out a strategy which guarantees that at least 999 of them remain
in the Council?

\item\emph {2012 FSA2.} Chip and Dale play the following game. Chip starts by splitting 1001 nuts between three piles, so Dale can see it. In response, Dale chooses some number $N$ from 1 to 1001. Then Chip moves nuts from the piles he prepared to a new (fourth) pile until there will be exactly $N$ nuts in any one or more piles.  When Chip accomplishes his task, Dale gets an exact amount of nuts that Chip moved. What is the maximal number of nuts that Dale can get for sure, no matter how Chip acts? (Naturally, Dale wants to get as many nuts as possible, while Chip wants to lose as little as possible).

\item\emph {2012 FSA5.} Peter and Paul play the following game. First, Peter chooses some positive integer $a$
with
the sum of its digits equal to 2012. Paul wants to determine this number; he knows only that the sum of the digits of Peter’s number is 2012. On each of his moves Paul chooses a positive integer $x$ and Peter tells him the sum of the digits of
$|x-a|$. What is the minimal number of moves in which Paul can determine Peter’s number for sure?

\item\emph {2012 SSO5.} In an $8\times 8$ chessboard, the rows are numbers from 1 to 8 and the columns are labelled from a
to h.  In a two-player game on this chessboard, the  rst player has a White Rook which starts
on the square b2, and the second player has a Black Rook which starts on the square c4.  The
two  players  take  turns  moving  their  rooks.   In  each  move,  a  rook  lands  on  another  square
in the same row or the same column as its starting square.  However, that square cannot be
under attack by the other rook, and cannot have been landed on before by either rook.  The
player without a move loses the game.  Which player has a winning strategy?

\item\emph {2011 FSO2.} Peter buys a lottery ticket on which he enters an $n$-digit number, none of the digits being 0. On the draw date, the lottery administrators will reveal an $n\times n$ table, each cell containing one of the digits from 1 to 9.  A ticket wins a prize if it does \emph{not} match any row or column of this table, read in either direction.  Peter wants to bribe the administrators to reveal the digits on some cells chosen by Peter, so that Peter can guarantee to have a winning ticket.
What is the minimum number of digits Peter has to know?

\item\emph {2011 SSA7.} Among a group of programmers, every two either know each other or do not know each other.
Eleven of them are geniuses.  Two companies hire them one at a time, alternately, and may
not  hire  someone  already  hired  by  the  other  company.   There  are  no  conditions  on  which
programmer a company may hire in the  rst round.  Thereafter, a company may only hire a
programmer who knows another programmer already hired by that company.  Is it possible
for the company which hires second to hire ten of the geniuses,  no matter what the hiring
strategy of the other company may be?

\item\emph {2010 FSO3.}  From a police station situated on a straight road in nite in both directions, a thief has stolen
a police car.  Its maximal speed equals 90$\%$ of the maximal speed of a police cruiser.  When
the theft is discovered some time later, a policeman starts to pursue the thief on a cruiser.
However, he does not know in which direction along the road the thief has gone, nor does he
know how long ago the car has been stolen.  Is it possible for the policeman to catch the thief?

\item\emph {2010 FSA4.} Two dueling wizards are at an altitude of 100 above the sea.  They cast spells in turn, and
each spell is of the form "decrease the altitude by $a$ for me and by $b$for my rival" where $a$ and $b$ are real numbers such that $0<a< b$.  Different spells have different values for $a$ and $b$.  The
set of spells is the same for both wizards, the spells may be cast in any order, and the same
spell may be cast many times.  A wizard wins if after some spell, he is still above water but
his rival is not.  Does there exist a set of spells such that the second wizard has a guaranteed
win, if the number of spells is (a) finite; (b) infinite.
\end{enumerate}

\section {IMO and shortlist}
\begin{enumerate}
\item\textbf {IMO 2012, Problem 3.} The liar's guessing game is a game played between two players $A$ and $B$. The rules of the game depend on two positive integers $k$ and $n$ which are known to both players.

At the start of the game $A$ chooses integers $x$ and $N$ with $1 \le x \le N.$ Player $A$ keeps $x$ secret, and truthfully tells $N$ to player $B$. Player $B$ now tries to obtain information about $x$ by asking player $A$ questions as follows: each question consists of $B$ specifying an arbitrary set $S$ of positive integers (possibly one specified in some previous question), and asking $A$ whether $x$ belongs to $S$. Player $B$ may ask as many questions as he wishes. After each question, player $A$ must immediately answer it with yes or no, but is allowed to lie as many times as she wants; the only restriction is that, among any $k+1$ consecutive answers, at least one answer must be truthful.

After $B$ has asked as many questions as he wants, he must specify a set $X$ of at most $n$ positive integers. If $x$ belongs to $X$, then $B$ wins; otherwise, he loses. Prove that:

1. If $n \ge 2^k,$ then $B$ can guarantee a win.
2. For all sufficiently large $k$, there exists an integer $n \ge (1.99)^k$ such that $B$ cannot guarantee a win.

\item\emph {2015 C4.} Let $n$ be a positive integer. Two players $A$ and $B$ play a game in which they take turns choosing positive integers $k \le n$. The rules of the game are:

(i) A player cannot choose a number that has been chosen by either player on any previous turn.
(ii) A player cannot choose a number consecutive to any of those the player has already chosen on any previous turn.
(iii) The game is a draw if all numbers have been chosen; otherwise the player who cannot choose a number anymore loses the game.

The player $A$ takes the first turn. Determine the outcome of the game, assuming that both players use optimal strategies.

\item\emph {2014 C8.} A card deck consists of $1024$ cards. On each card, a set of distinct decimal digits is
written in such a way that no two of these sets coincide (thus, one of the cards is empty). Two players alternately take cards from the deck, one card per turn. After the deck is empty, each player checks if he can throw out one of his cards so that each of the ten digits occurs on an even number of his remaining cards. If one player can do this but the other one cannot, the one who can is the winner; otherwise a draw is declared.
Determine all possible first moves of the first player after which he has a winning strategy.

\item\emph {2013 C8.} Players $A$ and $B$ play a "paintful" game on the real line. Player $A$ has a pot of paint with four units of black ink. A quantity $p$ of this ink suffices to blacken a (closed) real interval of length $p$. In every round, player $A$ picks some positive integer $m$ and provides $1/2^m $ units of ink from the pot. Player $B$ then picks an integer $k$ and blackens the interval from $k/2^m$ to $(k+1)/2^m$ (some parts of this interval may have been blackened before). The goal of player $A$ is to reach a situation where the pot is empty and the interval $[0,1]$ is not completely blackened.
Decide whether there exists a strategy for player $A$ to win in a finite number of moves.

\item\emph {2012 C4.} Players $A$ and $B$ play a game with $N \geq 2012$ coins and $2012$ boxes arranged around a circle. Initially $A$ distributes the coins among the boxes so that there is at least $1$ coin in each box. Then the two of them make moves in the order $B,A,B,A,\ldots $ by the following rules:
(a) On every move of his $B$ passes $1$ coin from every box to an adjacent box.
(b) On every move of hers $A$ chooses several coins that were not involved in $B$'s previous move and are in different boxes. She passes every coin to and adjacent box.
Player $A$'s goal is to ensure at least $1$ coin in each box after every move of hers, regardless of how $B$ plays and how many moves are made. Find the least $N$ that enables her to succeed.

\item\emph {2009 C1.} Consider $2009$ cards, each having one gold side and one black side, lying on parallel on a long table. Initially all cards show their gold sides. Two player, standing by the same long side of the table, play a game with alternating moves. Each move consists of choosing a block of $50$ consecutive cards, the leftmost of which is showing gold, and turning them all over, so those which showed gold now show black and vice versa. The last player who can make a legal move wins.\\
(a) Does the game necessarily end?\\
(b) Does there exist a winning strategy for the starting player?

\item\emph {2009 C5.} Five identical empty buckets of $2$-liter capacity stand at the vertices of a regular pentagon. Cinderella and her wicked Stepmother go through a sequence of rounds: At the beginning of every round, the Stepmother takes one liter of water from the nearby river and distributes it arbitrarily over the five buckets. Then Cinderella chooses a pair of neighbouring buckets, empties them to the river and puts them back. Then the next round begins. The Stepmother goal's is to make one of these buckets overflow. Cinderella's goal is to prevent this. Can the wicked Stepmother enforce a bucket overflow?
\end{enumerate}

\section {JOM and shortlist}
\begin{enumerate}
\item\textbf {JOM 2015, Problem 5.} Navi and Ozna are playing a game where Ozna starts first and the two take turn making moves. A positive integer is written on the board. A move is to (i) subtract any positive integer at most 2015 from it
or (ii) given that the integer on the board is divisible by 2014, divide by 2014. The first person to make the
integer 0 wins. To make Navi's condition worse, Ozna gets to pick integers $a$ and $b$, $a\ge 2015$ such that all
numbers of the form $an + b$ will not be the starting integer, where n is any positive integer.
Find the minimum number of starting integer where Navi wins.

\item\textbf {JOM 2013, Problem 4.} Let $n$ be a positive integer. A \emph{pseudo-Gangnam Style} is a dance competition between players $A$ and $B$. At time $0$, both players face to the north. For every $k\ge 1$, at time $2k-1$, player $A$ can either choose to stay stationary, or turn $90^{\circ}$ clockwise, and player $B$ is forced to follow him; at time $2k$, player $B$ can either choose to stay stationary, or turn $90^{\circ}$ clockwise, and player $A$ is forced to follow him.

\

After time $n$, the music stops and the competition is over. If the final position of both players is north or east, $A$ wins. If the final position of both players is south or west, $B$ wins. Determine who has a winning strategy when:

\

(a) $n=2013^{2012}$;

(b) $n=2013^{2013}$.

\item\emph {2015 C1.} Baron and Peter are playing a game. They are given a simple finite graph $G$ with $n\ge 3$ vertex and $k$
edges that connects the vertices. First Peter labels two vertices $A$ and $B$, and places a counter at $A$. Baron
starts first. A move for Baron is move the counter along an edge. Peter's move is to remove an edge from
the graph. Baron wins if he reaches $B$, otherwise Peter wins.
Given the value of $n$, what is the largest $k$ so that Peter can always win?

\item\emph {2014 C1.} There are 8 white rooks and 8 black rooks on the chessboard. The white rooks are on the rst rank
while the black ones are on the last rank. Each player plays 3 moves per turn. In every move, one can
either move one of his piece, provided that the move is legal (rooks moves horizontally, bishops moves
diagonally.), or place a bishop of his color and put it on an unoccupied square of the chessboard. Prove
that the rst player has a non-losing strategy. (Note : A player wins if his opponents have no possible
moves.)

\item\emph {2014 C3.} Ivan and Avin play a game. Ivan has 1 blue stone while Avin has 11 red stones and 7 blue stones at the
begining. At each turn, they can perform one of the following operations on the stones they have:\\
(i) add or remove any number of stones of the same colour.\\
(ii) add or remove same number of stones from both colours.\\
(iii) change the colour of some of their stones.\\
Ivan moves first. Ivan wins if he can obtain the same number of red stones and blue stones as Avin have
at that time within $2014^{2014}$ moves. Otherwise, Avin wins. Who has a winning strategy?

\item\emph {2014 C7.} $A$ and $B$ play the game Kazy, with $A$'s turn first. In Kazy, there are $n$ rows with $i$ vertical lines on the $i$-th row for $1\le  i\le  n$, where $n$ is a natural number. The 2 players take turns to cross out consecutive
blocks of lines (lines that are on the same row and not separated by other lines). The player who cross
out the last dot loses. For which natural number $n$ does A has a winning strategy?

\item\emph {2013 C2.} Let there be $2n$ cards lying on the table with a positive integer (not necessarily distinct) written on each
card. Two players take turns taking cards from either end of the table and the player with the greatest
sum from all the integers on his cards wins. Prove that the first player can always win or draw the game.

\item\emph {2013 C6.} Let $N$ be a natural number, which is known to $A, B$ and $C$. First $C$ randomly chooses a natural number
x$x\in [1,N]$. Starting with $A, A$ and $B$ take turns guessing $x$. In each turn, a player chooses a number
in the possible range of $x$. If the guess is incorrect, $C$ will announce whether $x$ is greater than or less
than the guessing. Then the next player chooses a number in the new range of $x$, and so on until one of
the players guesses the correct number. Assuming that both $A$ and $B$ are very smart and both of them
want to win the game, what is the probability that $A$ wins if\\
(a) the player who guesses the correct number wins?\\
(b) the player who guesses the correct number loses?
\end{enumerate}

\end{document}