\documentclass[11pt,a4paper]{article}
\usepackage{amsmath, amssymb, fullpage, mathrsfs, bm, pgf, tikz}

\begin{document}

\title{Linear recurrences}
\author{Anzo Teh Zhao Yang}
\date{24 July 2016}
\maketitle

Consider the following sequence:\\
$a_0=1, a_1=1, a_n=a_{n-1}+a_{n-2}, \forall n\ge 2.$

The problem might be generous enough to give you the closed form and ask you to prove it (which is very very easy), but this is, often, not the case. Finding the genral term \emph{ourselves} can be very difficult when the pattern isn't really clear. Fortunately, we have two 'cracks' here: generating functions and matrices.

\section {Generating functions.}
Consider the function $F(x)=a_0+a_1x+a_2x^2+\cdots +a_nx^n+\cdots $ (i.e. infinite series). If we know how $F(x)$ looks like, we should probably know how each of the coefficients behave.

Take the same sequence above, and we know that $F(x)=1+x+(a_0+a_1)x^2+(a_1+a_2)x^3+\cdots$=$1+x+(a_0x^2+a_1x^3+\cdots)+(a_1x^2+a_2x^3+\cdots)$=$1+x+x^2(a_0+a_1x+\cdots)+x(a_1x+a_2x^3+\cdots)$=$1+x+x^2F(x)+x(F(x)-1)$. This is equivalent to $F(x)(1-x-x^2)=1$, or $F(x)=(1-x-x^2)^{-1}$. Now we don't seem to be able to proceed as we have an inverse of a trinomial, but fortunately we have the following tactics:
\begin{enumerate}
\item Partial fractions.\\
This is a way by which we write a rational function (i.e. quotient of two polynomials) with numerator of degree at most 1 and denominator of degree 2, in terms of two rational functions, each with constant numerator and linear denominator:
$$\dfrac{ax+b}{cx^2+dx+e}=\dfrac{f}{gx+h}+\dfrac{i}{jx+k}$$
(Condition: the quadratic $cx^2+dx+e$ must not have repeated roots; otherwise, we have to write the partial fraction as $\dfrac{A}{(x-k)^2}+\dfrac{B}{x-k}$.)

Now we need to factorize our quadratic, $1-x-x^2$ first, which ends us up with weird terms anyway. For some reasons (which we will show later) it is better to factorize it in terms of $(1-\alpha x)(1-\beta x)$, whereby $\alpha+\beta=1, \alpha\beta=-1$. We proceed to the following: $$\dfrac{1}{1-x-x^2}=\dfrac{A}{1-\alpha x}+\dfrac{B}{1-\beta x}$$.
Equating coefficients yield $A+B=A\beta+B\alpha =1$, and we can use the technique of solving simultaneous eqautions to get $A=\frac{1-\alpha}{\beta-\alpha}$ and $B=\frac{\beta -1}{\beta-\alpha}$.

Still, we need to find the expansion of $\dfrac{A}{1-\alpha x}$ and $\dfrac{B}{1-\beta x}$. Remember that $(a+b)^{n}=a^n+\dbinom{n}{1}a^{n-1}b+\cdots +b^n$? There's a (generous) generalization to any $n$, integer or not:

\item Binomial formula
$(1+x)^k=1+\displaystyle\sum_{n=1}^{\infty} \left(\frac{\displaystyle\prod_{i=0}^{n-1} (k-i)}{n!}x^n\right)$ for $k\in\mathbb{Q}$.\\
In particular, it is always true that $\frac{1}{1-a}=1+a+a^2+\cdots$. Therefore, we have the following: $$\dfrac{A}{1-\alpha x}+\dfrac{B}{1-\beta x}=\frac{1-\alpha}{\beta-\alpha}(1+\alpha x+(\alpha x)^2+\cdots)+ \frac{\beta -1}{\beta-\alpha} (1+\beta x+(\beta x)^2+\cdots )$$.

Recall that the $n$-th is the coefficient of $x^n$, so this is $\dfrac{1-\alpha}{\beta-\alpha}(\alpha ^n)$ + $\dfrac{\beta -1}{\beta-\alpha} (\beta ^n)$ = $\dfrac{\beta ^{n+1}-\beta^{n}-\alpha^{n+1}+\alpha^{n}}{\beta-\alpha}$. Also notice that $\alpha$ and $\beta$ are roots of equation $x^2-x-1=0, $ so $\alpha=\dfrac{1-\sqrt{5}}{2}$ and $\beta=\dfrac{1+\sqrt{5}}{2}$; moreover $\alpha ^{2}=\alpha+1$ (same for $\beta$), so $\dfrac{\beta ^{n+1}-\beta^{n}-\alpha^{n+1}+\alpha^{n}}{\beta-\alpha}$ = $\dfrac{\beta ^{n-1}-\alpha^{n-1}}{\beta-\alpha}$ = $\dfrac{1}{\sqrt{5}}\left(\left(\dfrac{1+\sqrt{5}}{2}\right)^{n-1}-\left(\dfrac{1-\sqrt{5}}{2}\right)^{n-1}\right)$. $\blacksquare$

Of course, this isn't exactly the Fibonacci sequence, but good thing is that this sequence has 0 as its 0-th term and 1 as 1st term. The $n$-th term, $F_{n}$, is simply $\dfrac{1}{\sqrt{5}}\left(\left(\dfrac{1+\sqrt{5}}{2}\right)^{n}-\left(\dfrac{1-\sqrt{5}}{2}\right)^{n}\right)$.
\end{enumerate}

We name $\alpha$ and $\beta$ the characteristic roots of this recurrence relation (provided it's linear). Things get more complicated when a term is defined on previous $k$ terms where $k>2$, but in special cases the same theory can be used. Moreover, knowing the explicit formula gives rise of the following:
\begin {enumerate}
\item $\alpha^n+\beta^n$ is an integer for all $n\ge 0$.
\item $F_{n}\mid F_{kn}$ for all $k, n\in\mathbb{N}$.
\item Let $G_{n}=\frac{F_{2n}}{F_n}$, $\forall n\ge 1$, then $G_{n+2}=G_{n}+G_{n+1}$, $\forall n\ge 1$.
\end{enumerate}

\section {Matrices.}
OK let's move on to the next topic: matrices. But this topic is rather loaded.\\
\textbf{Definition.} A $m\times n$ matrix is a rectangle of $m$ rows and $n$ columns, each entry containing a complex number. We say $m\times n$ as its \emph{dimension.}

The appearance of matrices are like the following:\\
 \[\left( \begin{array}{cccc}
a_{11} \quad a_{12} \quad \ldots \quad a_{1n}\\
a_{21} \quad a_{22} \quad \ldots \quad a_{2n}\\
\vdots\\
a_{m1} \quad a_{m2} \quad \ldots \quad a_{mn}\\
\end{array} \right)\]

\subsection {Special types of matrices.}
\begin{enumerate}
\item Square matrix.\\
This type of matrix has $m\times n$, i.e. the number of rows and the number of columns are the same.

\item Row matrix.\\
This type of matrix has only one row, in the form  $\left( \begin{array}{c}
a_{11} \quad a_{12} \quad \ldots \quad a_{1n}\\
\end{array} \right)$.

\item Column matrix.\\
This type of matrix has only one column, in the form  $\left( \begin{array}{cccc}
a_{11}\\
a_{21}\\
\vdots\\
a_{m1}\\
\end{array} \right)$.

The following listing are for square matrices only.

\item Triangular matrix.\\
An upper triangular matrix has nonzero entries only at entries on or above its main diagonal; lower triangular matrix is defined similarly. Below shows the example of upper (left) and lower (right) triangular matrices.

$$\left(\begin{array}{cccc}
a_{11} & a_{12} & \ldots & a_{1n}\\
0 & a_{22} & \ldots & a_{2n}\\
&\vdots\\
0 & 0 & \ldots & a_{nn}\end{array}\right)\qquad
\left(\begin{array}{cccc}
a_{11} & 0 & \ldots & 0\\
a_{21} & a_{22} & \ldots & 0\\
&\vdots\\
a_{n1} & a_{n2} & \ldots & a_{nn}\end{array}\right)$$.

\item Diagonal matrix (square matrix only).\\
This is a special matrix that has 0 for entries not along its main diagonal. A matrix is both upper and lower triangular iff it is diagonal matrix.

 \[\left( \begin{array}{cccc}
a_{11} \quad 0 \quad \ldots\quad 0\\
0 \quad a_{22} \quad \ldots\quad 0\\
\vdots\\
0 \quad 0 \quad \ldots\quad a_{nn}\\
\end{array} \right)\]

\item Identity matrix, denoted as $I$ (square matrix only).\\
This is a subset of diagonal matrices, only with extra constraint that every entry on its main diagonal is 1. For any matrix $A$, $AI=IA=A$ (provided $AI$, $IA$ are defined).

\end{enumerate}

\subsection{Theory and operation behind matrices.}
\begin{enumerate}
\item Addition of matrices.\\
This is simple: if $A$ and $B$ are matrices with $ij$-th entry as $a_{ij}$ and $b_{ij}$, then the $ij$-th entry of $A+B$ is just $a_{ij}+b_{ij}$. The only constraint is, $A$ and $B$ must have the same dimension (otherwise $A+B$ is undefined).

\item Scalar multiplication of matrices.
If $k$ is a real number and $A$ is a matrix with entries $a_{ij}$, then the $ij$-th entry of $kA$ is $k\cdot a_{ij}$. If $k=0$ then the resulting matrix has all entries 0; we call this a zero matrix (sometimes we just denote it as 0).

\item Multiplication of two matrices.\\
This is far less straightforward compared to addition or scalar multiplication. If $C=AB$ then $c_{ij}=a_{i1}b_{1j}+a_{i2}b_{2j}+\cdots a_{im}b_{mj}$, where $m$ is both the number of columns of $A$ and number of rows of $B$.

Notice that if $A$ and $B$ have dimensions $m\times n$ and $p\times q$ then $AB$ is defined iff $n=p$, and the resulting $AB$ has dimension $m\times q$ (as suggested by above), and in general $AB\neq BA$, i.e. not communicative. Worse still, there are cases where $AB$ is defined but not $BA$. However, multplication of matrices is associative, i.e. $(AB)C=A(BC)$, provided that $AB$ and $BC$ are defined.

\item Determinant of square matrix.\\
For $1\times 1$ matrix, the determinant is the only entry in the matrix. For $n>1$, the determinant of $A$, an $n\times n$ matrix is defined (inductively) as below:

Let $C_{ij}$ be the $(n-1)\times (n-1)$ matrix defined by removing the $i$-th row and $j$-th column of $A$ (without changing the positions of the entries), then the determinant of $A$ is $a_{11}\det (C_{11})-a_{12}\det (C_{12})-\cdots +(-1)^{n-1}a_{1n}\det (C_{1n})$. 

Notice also that if $\sigma$ is a permutaion of $\{1,2,\cdots n\}$ and $I_{\sigma}$ is $(-1)^{\sigma} (a_{1\sigma (1)}+a_{2\sigma(2)}+\cdots +a_{n\sigma(n)})$ then the determinant is the sum of all $n!$ such sigmas.

Some useful identities:\\
$\bullet \det A\det B=\det (AB)$, for $A$ and $B$ same dimension.\\
$\bullet$ The determinant of any triangular matrix is exactly the product of entries on its main diagonal. In particular, this is true for diagonal matrices, and we infer that the determinant of an identity matrix is 1.

\item Inverse of a matrix (square matrix with nonzero determinant).\\
For any matrix $A$ with $\det (A)\neq 0$, there is a unique matrix $B$ such that $AB$ and $BA$ are both the identity matrix.\\
Uniqueness: suppose that $AB=BA=AC=CA=I$, then $B=BI=B(AC)=(BA)C=IC=C$.\\
Existence is harder to prove, but if $\det (A)=0$ then contradiction can be achieved by the fact $\det A\det B=\det AB$.

The inverse of a matrix is, in general, hard to find. However, for $2\times 2$ matrix it is rather straightforward: the inverse of $\left(\begin{array}{cc}
a\quad b\\
c\quad d\\
\end{array}\right)$ is $\dfrac{1}{ad-bc}\left(\begin{array}{cc}
d & -b\\
-c & a\\
\end{array}\right)$.

Something harder to believe: taking $A$ as square matrix again, and $x$ a matrix s.t. $Ax$ is defined. If $\det A\neq 0$, then $Ax=0$ iff $x=0$; conversely, if $\det A=0$ then there exists a non-zero matrix $x$ such that $Ax=0$.

\item Eigenvectors and eigenvalues.\\
For a matrix $A$, if $Ax=kx$ for some non-zero vector $x$, then $x$ is an eigenvector of $A$, with corresponding eigenvalue $k$. For convenience let's assume that $x$ has only 1 column.

The technique to find eigenvectors and eigenvalues isn't practical for $A$ with dimension $\ge 4\times 4$, but doable for $2\times 2$ and sometimes $3\times 3$ matrices. Notice that $Ax=kx=kIx$ so $(A-kI)x=0$, meaning that $\det (A-kI)=0.$ We present the method for $2\times 2$ whereby $A-kI$ is\\
 \[\left( \begin{array}{cc}
a_{11} & a_{12}\\
a_{21} & a_{22}\\
\end{array} \right)-
 \left( \begin{array}{cc}
k & 0\\
0 & k\\
\end{array} \right)=
 \left( \begin{array}{cc}
a_{11}-k & a_{12}\\
a_{21} & a_{22}-k\\
\end{array} \right)\]
The determinant is $(a_{11}-k)(a_{22}-k)-a_{12}a_{21}$, a quadratic equation in $k$. Once we located the roots of equation (say $k_1$ and $k_2$), we can proceed to find scalars $x_1$, $x_2$ such that $(a_{11}-k_1)x_1+a_{12}x_2=0$, so $\left( \begin{array}{cc}
x_1\\
x_2\\
\end{array} \right)$ is the eigenvalue for this $k_1$. (Similar method for $k_2$).

\item Diagonalization of matrices.\\
Given a square matrix $A$, we wish to write it in $A=PDP^{-1}$ where $D$ is a diagonal matrix.

Notice the benefit of doing so. Suppose you are given matrix $A$, say, $\left(\begin{array}{cc}
21\quad 22\\
31\quad 18\\
\end{array} \right)$, and you are to find $A^{1000}$. Virtually impossible right? Now $(PDP^{-1})^n=PD^{n}P^{-1}$ (convince yourself that this is true!) and since $D$ is diagonal, $ii$-th entry of $D^n$ is precisely $d_{ii}^n$ and $ij$-th entry 0. Easily determinable, right?

To diagonalize, we need to find the eigenvalue and eigenvectors of $A$ first. Let $x_1, x_2, \cdots x_n$ be the eigenvectors with corresponding eigenvalues $k_1, k_2,\cdots k_n$, then line up $x_1, x_2, \cdots x_n$ in that order to find $P$, and obtain $D$ by filling $k_i$ at the $ii$-th entry.
\end{enumerate}

\subsection{Problem solving}
OK so done with the theory, but how to use this to solve recurrence relations? Again we use quadratics (i.e. each term is defined on previous two terms) to simplify matters. This time, we tackle $a_{k+2}=a_{k+1}+6a_{k}$, with $a_0$ and $a_1$ specified, to avoid the messy square roots again (as in Fibonacci).

Denote $v_k=\left( \begin{array}{cc}
a_k\\
a_{k+1}\\
\end{array}\right)$,
then 
$v_{k+1}$ = $\left( \begin{array}{cc}
a_{k+1}\\
a_{k+2}\\
\end{array}\right)$
=$\left( \begin{array}{cc}
a_{k+1}\\
a_{k+1}+6a_k\\
\end{array}\right)$
=$\left(\begin{array}{cc}
0\quad 1\\
6\quad 1\\
\end{array}\right)$
$\left( \begin{array}{cc}
a_{k}\\
a_{k+1}\\
\end{array}\right)$.
Denote $\left(\begin{array}{cc}
0\quad 1\\
6\quad 1\\
\end{array}\right)$ as $A$, then we can see, inductively, then $v_k=A^kv_0$. The onus, again, lies on the diagonalization of $A$, so we have: $\det \left(\begin{array}{cc}
-x & 1\\
6 & 1-x\\
\end{array}\right)$=$x^2-x-6$=0. Easily seen, that, $x=3$ or $-2$. For the first case, we need 
$\left(\begin{array}{cc}
-3 & 1\\
6 & -2\\
\end{array}\right)$
$\left(\begin{array}{cc}
p\\
q\\
\end{array}\right)=0$, which we can tak $p=1$ and $q=3$. For the second case we need 
$\left(\begin{array}{cc}
2\quad 1\\
6\quad 3\\
\end{array}\right)$
$\left(\begin{array}{cc}
r\\
s\\
\end{array}\right)=0$, so $r=1$ and $s=-2$ (there are infinitely many values for $p, q, r, s$; each of them work well for diagonalization purposes

Now, we need $P=\left(\begin{array}{cc}
p\quad r\\
q\quad s\\
\end{array}\right)=
\left(\begin{array}{cc}
1 & 1\\
3 & -2\\
\end{array}\right)$, $D=\left(\begin{array}{cc}
3 & 0\\
0 & -2\\
\end{array}\right)$. Now $P^{-1}=\dfrac{1}{-5}\left(\begin{array}{cc}
-2 & -1\\
-3 & 1\\
\end{array}\right)$. It is therefore no longer difficult to see that $A^n$ = $PD^nP^{-1}$ = $\dfrac{1}{-5}\left(\begin{array}{cc}
1 &1\\
3 & -2\\
\end{array}\right)$
$\left(\begin{array}{cc}
3^n & 0\\
0 & (-2)^n\\
\end{array}\right)$
$\left(\begin{array}{cc}
-2 & -1\\
-3 & 1\\
\end{array}\right)$ = $\left(\begin{array}{cc}
-2\cdot3^n & -(3^n)+(-2)^n\\
(-2)\cdot 3^{n+1}-3\cdot (-2)^{n+1} & -(3^{n+1})+(-2)^{n+1}\\
\end{array}\right).$ Simply multiply this by $v_0$ we get the recurrence relation we want.

In general, the relation $a_k=pa_{k-1}+qa_{k-2}$ has $A=$ $\left(\begin{array}{cc}
0\quad 1\\
q\quad p\\
\end{array}\right)$ and in finding its diagonalization we need $\det$ $\left(\begin{array}{cc}
-x\quad 1\\
q\quad p-x\\
\end{array}\right)$ to be 0, which reduces to the quadratic equation $x^2-px-q=0$. If the roots are $\alpha$ and $\beta$, then their respective eigenvectors are $\left( \begin{array}{cc}
1\\
\alpha\\
\end{array}\right)$ and $\left( \begin{array}{cc}
1\\
\beta\\
\end{array}\right)$. Therefore let's put $P=$ $\left(\begin{array}{cc}
1\quad 1\\
\alpha\quad \beta\\
\end{array}\right)$ with $P^{-1}$ as $\dfrac{1}{\beta-\alpha}$ $\left(\begin{array}{cc}
\beta\quad -1\\
-\alpha\quad 1\\
\end{array}\right)$. Bear in mind, however, that this doesn't work for $A$ with two equal eigenvalues (e.g. when we have $a_k=2a_{k-1}-a_{k-2}$).

\section{Conclusion.}
By now should have seen the similarity between the two: parallelism between characteristic roots! In short, generating function is more elementary, but matrices offer more flexibility particularly when $a_0$ and $a_1$ are not explicitly given.

Some practice problems are really useful:
\begin{enumerate}
\item \emph{JOM 2015, A6.} Let $(a_{n})_{n\ge 0}$ and $(b_{n})_{n\ge 0}$ be two sequences with arbitrary real values $a_0, a_1, b_0, b_1$. For $n\ge 1$, let $a_{n+1}, b_{n+1}$ be defined in this way:
$$a_{n+1}=\dfrac{b_{n-1}+b_{n}}{2}, b_{n+1}=\dfrac{a_{n-1}+a_{n}}{2}$$
Prove that for any constant $c>0$ there exists a positive integer $N$ s.t. for all $n>N$, $|a_{n}-b_{n}|<c$.\\
Outline: considering the sequence $c_n=a_n-b_n$ then it fulfills the relation $2c_{n+1}+c_n+c_{n-1}=0$, and the characteristic roots are $-\frac14\pm\frac{\sqrt{7}}{4}i$. It is not hard to verify that each of them has modulus of only $\frac 12$, hence must converge to 0.

\item\emph {IMO 2013, A1.} Let $n$ be a positive integer and let $a_1, \ldots, a_{n-1} $ be arbitrary real numbers. Define the sequences $u_0, \ldots, u_n $ and $v_0, \ldots, v_n $ inductively by $u_0 = u_1 = v_0 = v_1 = 1$, and $u_{k+1} = u_k + a_k u_{k-1}$, $v_{k+1} = v_k + a_{n-k} v_{k-1}$ for $k=1, \ldots, n-1.$

Prove that $u_n = v_n.$

\item\emph{IMO 2004, N4.} Let $k$ be a fixed integer greater than 1, and let ${m=4k^2-5}$. Show that there exist positive integers $a$ and $b$ such that the sequence $(x_n)$ defined by \[x_0=a,\quad x_1=b,\quad x_{n+2}=x_{n+1}+x_n\quad\text{for}\quad n=0,1,2,\dots,\] has all of its terms relatively prime to $m$.

\end{enumerate}

Now we are ready to move to this problem. It's difficult (being the last NT problem on the shortlist), but the above definitely helps a lot.

\emph {IMO 2010, N6.} The rows and columns of a $2^n \times 2^n$ table are numbered from $0$ to $2^{n}-1.$ The cells of the table have been coloured with the following property being satisfied: for each $0 \leq i,j \leq 2^n - 1,$ the $j$-th cell in the $i$-th row and the $(i+j)$-th cell in the $j$-th row have the same colour. (The indices of the cells in a row are considered modulo $2^n$.) Prove that the maximal possible number of colours is $2^n$.

\textbf {Solution.} Notice that each cell $(i,j)$ has the same colour as $(j,i)$, so let's consider the sequence $i,j,i+j,i+2j, 2i+3j, 3i+5j,\cdots $. The cell with coordinates exactly appearing as two consecutive terms in that order has the same colour as $(i,j)$, and moreover if $i$ is the 0th term then the $n$-th term is $F_{n-1}i+F_{n}j$ (where $F$ is Fibonacci sequence; by convention we put $F_{-1}=1$.)

Notice that the Fibonacci sequence, when considered modulo $m$ ($m$ any positive integer), is periodic. Indeed, take the pairs $P_{i}=(F_{i}, F_{i+1})$ we know that among $P_0, P_1,\cdots P_{m^2}$ two of them are equal (say $P_h$ and $P_k$). It follows that $F_{h+2}$ and $F_{k+2}$ are equal mod $m$ too, since $F_{h+2}=F_{h}+F_{h+1}\equiv F_{k}+F_{k+1}=F_{k}\pmod {m},$ which follows (inductively) that it has period $k-h$. Notice, also, that, if $i$ and $j$ are not both even, and the sequence $a_n=F_{n-1}i+F_nj$ mod $2^n$ has period $m$, then $i(F_{m-1}-F_{-1})+j(F_{m}-F_{0})\equiv i(F_{m}-F_{0})+j(F_{m+1}-F_{1})\equiv 0\pmod {2^{n}}.$ It is not hard to verify that this happens iff $F_m\equiv F_0$ and $F_{m+1}\equiv F_1$. Indeed, the sequence $D_n=F_{m+n}-F_n$ also follows $D_{n+2}=D_{n+1}+D_n$. Now consider the highest power of 2, namely $x$, dividing $D_0, D_1$ and $D_{-1}$. Then two of them is divisible exactly by $2^x$ and one of them divisible by $2^{x+1}$.  Now suppose that $iD_{-1}+jD_0$ and $iD_0+jD_1$ are both divisible by $2^{x+1}$. If $D_{0}$ divisible by $2^{x+1}$ then we must have $i, j$ both even (contradiction) since $2^{x+1}$ must divide both $iD_{-1}$ and $jD_{1}$. If $D_{-1}$ divisible by $2^{x+1}$ then $j$ must be even to have $2^{x+1}\mid jD_0$ and subsequently $2^{x+1}\mid iD_{-1}$, forcing $i$ even (contradiction). Hence $2^{x+1}$ cannot divide both these numbers, and we must have $x\ge n$ in order to have both  $iD_{-1}+jD_0$ and $iD_0+jD_1$ divisible by $2^n$.

With this, we proceed to the following lemma:\\
\emph {Lemma.} A Fibonacci sequence, when considered modulo $2^n$, has minimal period $3\cdot 2^{n-1}$.\\
Proof: we use the equivalence $F_n$ = $\dfrac{1}{\sqrt{5}}\left(\left(\dfrac{1+\sqrt{5}}{2}\right)^{n}-\left(\dfrac{1-\sqrt{5}}{2}\right)^{n}\right)$, and for simplicity name $\dfrac{1+\sqrt{5}}{2}$ as $\beta$ and $\dfrac{1-\sqrt{5}}{2}$ and $\alpha$. Let's consider sequences $G_{n}=\frac{F_{2n}}{F_n}$ and $H_{n}=G_n-F_n$, $\forall n\ge 1$. Now $G_n=\beta ^n+\alpha ^n$, and since $\beta$ and $\alpha$ are roots of $x^2=x+1$, we have $G_{n+2}=G_{n+1}+G_n$, $\forall n\ge 1$. The same happens to $H_n$ by its definition. Moreover, $H_n=\beta ^n+\alpha ^n-\dfrac{1}{\sqrt{5}}(\beta ^n-\alpha ^n)$ = $\dfrac{\sqrt{5}-1}{\sqrt{5}}\left(\dfrac{1+\sqrt{5}}{2}\right)\left(\dfrac{1+\sqrt{5}}{2}\right)^{n-1}$ + $\dfrac{1+\sqrt{5}}{\sqrt{5}}\left(\dfrac{1-\sqrt{5}}{2}\right)\left(\dfrac{1-\sqrt{5}}{2}\right)^{n-1}=2(\beta ^{n-1}-\alpha ^{n-1})$ = $2F_{n-1}.$

To prove the lemma, we need the following claim:\\
$\bullet 2^{n+1}\mid F_{3\cdot 2^{n-1}}$, $\forall n\ge 2.$\\
$\bullet F_{3\cdot 2^{n-1}+1}\equiv 2^{n}+1\pmod {2^{n+1}}$, $\forall n\ge 2.$

For the first claim, $n=2$ is obvious to see as $F_6=8$ is divisible by $2^{2+1}$. For inductive step we use the identity $F_{3\cdot 2^{n+1}}=F_{3\cdot 2^{n}}\cdot G_{3\cdot 2^{n}}$. We claim that $G_{3k}$ is even for all $k$. Indeed, if we investigate the sequence we have $G_1=1, G_2=3, G_3=4$ and we know that $G$ has period 3 in mod 2 (recall that $G$ gollows $G_{n+2}=G_{n+1}+G_n$). Therefore, with $F_{3\cdot 2^{n}}$ divisible by $2^{n+1}$ we must have $F_{3\cdot 2^{n+1}}$ divisible by $2^{n+2}$, and the induction hypothesis follows. To prove the second claim, we observe that $F_{3\cdot 2^{n-1}+1}$ = $F_{3\cdot 2^{n-1}+2}-F_{3\cdot 2^{n-1}}\equiv F_{3\cdot 2^{n-1}+2}$ = $F_{3\cdot 2^{n-2}+1}\cdot G_{3\cdot 2^{n-2}+1}$ = $F_{3\cdot 2^{n-2}+1}\cdot (F_{3\cdot 2^{n-2}+1}+H_{3\cdot 2^{n-2}+1})$ = $F_{3\cdot 2^{n-2}+1}^2$ + $2F_{3\cdot 2^{n-2}+1}\cdot F_{3\cdot 2^{n-2}}$ $\equiv$ $F_{3\cdot 2^{n-2}+1}^2$ $\pmod {2^{n+1}}$, since $H_{k+1}=2F_k$ and $2^n\mid F_{3\cdot 2^{n-1}}$. Finally, using induction hypothesis, we know that $F_7=13\equiv 5\pmod {8}$ for base case. As for inductive step we assume that $F_{3\cdot 2^{n-2}+1}\equiv 2^{n-1}+1\pmod {2^n}$ for some $n\ge 3$, then $F_{3\cdot 2^{n-2}+1}\equiv a\cdot 2^{n-1}+1\pmod {2^{n+1}}$ for some odd integer $a$. We then hve $F_{3\cdot 2^{n-1}+2}\equiv F_{3\cdot 2^{n-2}+1}^2\equiv (a\cdot 2^{n-1}+1)^2$ = $a^2\cdot 2^{2n-2}+a\cdot 2^n +1$ $\equiv 2^n+1$ $\pmod {2{n+1}}$, as $2n-2\ge n+1$ for $n\ge 3$. This completes the claim.

Now obviously $F_{3\cdot 2^{n-1}}\equiv 0\equiv F_0$ and $F_{3\cdot 2^{n-1}+1}\equiv 1= F_1$ modulo $2^n$, so $3\cdot 2^{n-1}$ is a period. To prove that this period is minimal, we use induction again. For $n=1$ the sequence in mod 2 is 0,1,1,0,1,1,$\cdots$ and for $n=2$ the sequence is 0,1,1,2,3,1,0,1,1,2,3,1,$\cdots$. Now suppose that the minimal period mod $2^n$ is $3\cdot 2^{n-1}$ for some $n\ge 2$. If $M$ is the minimal period of mod $2^{n+1}$ then it is a period od modulo $2^n$ as well, so $3\cdot 2^{n-1}\mid M$. From the fact that $3\cdot 2^n$ is a period of mod $2^{n+1}$ we have $M\in \{3\cdot 2^{n-1}, 3\cdot 2^n\}.$ However, from above $F_{3\cdot 2^{n-1}+1}\equiv 2^{n}+1\pmod {2^{n+1}}$, so $3\cdot 2^{n-1}$ cannot be a period, establishing the lemma. $\square$

To finish the solution, notice from above that for sequences with no consecutive terms which are both even, the minimal period if $3\cdot 2^{n-1}$. Since there are $\frac{3\cdot 2^{2n}}{4}=3\cdot 2^{2(n-1)}$ such grids, there must be exactly $\frac{3\cdot 2^{2(n-1)}}{3\cdot 2^{n-1}}=2^{n-1}$ sequences. For sequences not listed above, consecutive terms (and hence all numbers) are even, which can be reduced to the case of $2^{n-1}\times 2^{n-1}$ squares. Inductively this means we have $2^{n-1}+2^{n-1}=2^n$ sequences in total, and this is precisely the number of colours we are allowed to use. $\blacksquare$

\section {References.}
\begin{enumerate}
\item W. Keith Nicholson: 2013, \emph {Linear Algebra with Applications.} McGraw-Hill International Edition.
\item Yao Zhang: 2011, Mathematical Olympiad series, Vol.4, \emph {Combinatorial Problems in Mathematical Competitions.} World Scientific, East China Normal University Press.
\end{enumerate}

\subsection{Problem credits.}
\begin{enumerate}
\item IMO shortlist (2004, 2010, 2013).
\item Junior Olympiad of Mathematics 2015, shortlist.
\end{enumerate}
\end{document}