\documentclass[11pt,a4paper]{article}
\usepackage{amsmath, amssymb, fullpage, mathrsfs, bm, pgf, tikz}

\begin{document}

\title{Inversion}
\author{http://www.imomath.com/}
\date{24 June 2016}
\maketitle

\section {Introduction}
An inversion with respect to a given circle (sphere) is the map sending each point $A$  other than the center $O$  of the circle to the point $A'$  on ray $OA$  such that $OA'=\frac{r^2}{OA}$. What makes this map useful is the fact that it preserves angles and maps lines and circles onto lines or circles. Thus appropriate inversions can reduce the number of unpleasant circles (mapping them to lines) and often even turn a difficult problem into a quite simple one, as we show on a number of solved problems. Problems range from Ptolemy s inequality, to Feuerbach s theorem, and some of the hardest problems appearing on math competitions. 

\section {General properties}
Inversion $\Psi$ is a map of a plane or space without a fixed point $O$ onto itself, determined by a circle $k$ with center $O$  and radius $r$, which takes point $A\ne O$ to the point $A'=\Psi(A)$ on the ray $OA$  such that $OA'=\frac{r^2}{OA}$. From now on, unless noted otherwise, $X'$ always denotes the image of object $X$ under a considered inversion.

Clearly, map $\Psi$ is continuous and inverse to itself, and maps the interior and exterior of $k$ to each other, which is why it is called "inversion". The next thing we observe is that $\triangle P'OQ'\sim\triangle QOP$ for all points $P, Q$  (for $\angle POQ=\angle Q'OP'$  and $OP\prime /OQ\prime =(r^2/OP)/(r^2/OQ)=
OQ/OP$), with the ratio of similitude $\frac{r^2}{OP\cdot OQ}$. As a consequence, we have $$\angle OQ\prime P\prime = \angle OPQ\quad\mbox{and}
\quad P\prime Q\prime = \frac{r^2}{OP\cdot OQ}PQ.$$

What makes inversion attractive is the fact that it maps lines and circles into lines and circles. A line through $O$($O$ excluded) obviously maps to itself. What if a line $p$  does not contain $O$? Let $P$ be the projection of $O$ on $p$ and $Q\in p$ an arbitrary point of $p$. Angle $\angle OPQ=\angle OQ\prime P\prime$ is right, so $Q'$ lies on circle $k$ with diameter $OP'$. Therefore $\Psi(p)=k$ and consequently $\Psi(k)=p$. Finally, what is the image of a circle $k$  not passing through $O$? We claim that it is also a circle; to show this, we shall prove that inversion takes any four concyclic points $A,B,C,D$  to four concyclic points $A', B', C', D'$. The following angles are regarded as oriented. Let us show that $\angle A\prime C\prime B\prime =\angle A\prime D\prime B\prime$. We have $\angle A\prime C\prime B\prime =\angle OC\prime B\prime 
-\angle OC\prime A\prime =\angle OBC-\angle OAC$ and analogously $\angle
A\prime D\prime B\prime =\angle OBD-\angle OAD$, which implies  $\angle A\prime D\prime B\prime -\angle
A\prime C\prime B\prime =\angle CBD-\angle CAD=0$, as we claimed. To sum up:\\
$\bullet$ A line through $O$ maps to itself.\\
$\bullet$ A circle through $O$ maps to a line not containing $O$ and vice-versa.\\
$\bullet$ A circle not passing through $O$ maps to a circle not passing through $O$ (not necessarily the same). \\

Remark. Based on what we have seen, it can be noted that inversion preserves angles between curves, in particular circles or lines. Maps having this property are called conformal.

When should inversion be used? As always, the answer comes with experience and cannot be put on a paper. Roughly speaking, inversion is useful in destroying ``inconvenient" circles and angles on a picture. Thus, some pictures ``cry" to be inverted:

$\bullet$ There are many circles and lines through the same point $A$. Invert through $A$.

\textbf{Example}: Let $\Gamma_1$, $\Gamma_2$, $\Gamma_3$, $\Gamma_4$ be distinct circles such that $\Gamma_1$, $\Gamma_3$ are externally tangent at $P$, and $\Gamma_2$, $\Gamma_4$ are externally tangent at the same point $P$. Suppose that $\Gamma_1$ and $\Gamma_2$; $\Gamma_2$ and $\Gamma_3$; $\Gamma_3$ and $\Gamma_4$; $\Gamma_4$ and $\Gamma_1$ meet at $A$, $B$, $C$, $D$, respectively, and that all these points are different from $P$. Prove that

\[ \frac{AB\cdot BC}{AD\cdot DC}=\frac{PB^2}{PD^2}. \]

\textbf {Solution:} Consider the inversion of pole $ P$ and any power we want. Let $ F'$ be the image the figure $ F$ by the inversion (no matter what $ F$ is: circle, line, point etc.). We have $ \frac {T_{1}T_{2}}{T_{1}'T_{2}'} = \frac {PT_{2}}{PT_{1}'}$ and the analogous relations. From all of that we find $ \frac {T_{1}T_{2}\cdot T_{3}T_{2}}{T_{1}T_{4}\cdot T_{3}T_{4}} = \frac {T_{1}'T_{2}'\cdot T_{3}'T_{2}'}{T_{1}'T_{4}'\cdot T_{3}'T_{4}'}\cdot \frac {PT_{2}^{2}}{PT_{4}^{2}} = \frac {PT_{2}^{2}}{PT_{4}^{2}}$ because $ A_{1}'\parallel A_{3}',\ A_{2}'\parallel A_{4}'$, so $ T_{1}'T_{2}'T_{3}'T_{4}'$ is a parallelogram, so $ T_{1}'T_{2}' = T_{3}'T_{4}',\ T_{3}'T_{2}' = T_{1}'T_{4}'$. 

$\bullet$ There are many angles $AXB$ passing through fixed points $A$ and $B$. Invert through $A$ or $B$. 

\textbf {Example.} Let $ P$ be a point inside a triangle $ ABC$ such that
\[ \angle APB - \angle ACB = \angle APC - \angle ABC. \]
Let $ D$, $ E$ be the incenters of triangles $ APB$, $ APC$, respectively. Show that the lines $ AP$, $ BD$, $ CE$ meet at a point.

\textbf {Solution.} Apply an inversion with center at $A$ and radius $r$. Then the given condition becomes $\angle B\prime C\prime P\prime = \angle C\prime B\prime P\prime$, i.e. $B'P'=P'C'$. But $P\prime B\prime =\frac{r^2}{AP\cdot AB}PB$, so $AC/AB=PC/PB$. 

\section {Practice problems.}
\begin{enumerate}
\item Circles $k_1$, $k_2$, $k_3$, $k_4$ are such that $k_1$ and $k_3$ are each tangent to $k_2$ and $k_4$. Prove that the tangency points are either collinear or concyclic.

\item Prove that for any points $A, B,C, D$,$AB\cdot CD+AD\cdot BC\ge AC\cdot BD$, and that equality holds if and only if $A,B,C,D$ are on a circle or a line in this order. (Ptolemy s inequality)

\item Let $\omega$ be the semicircle with diameter $PQ$. A circle $k$ is tangent internally to $\omega$ and to segment $PQ$ at $C$. Let $AB$ be the tangent to $k$ perpendicular to $PQ$, with $A$ on $\omega$  and $B$ on segment $PQ$. Show that $AC$ bisects the angle $\angle PAB$. 

\item Points $A,B,C$ are given on a line in this order. Semicircles $\omega, \omega_1, \omega_2$ are drawn on  $AC,AB,BC$  respectively as diameters on the same side of the line. A sequence of circles $(k_n)$ is constructed as follows: $k_0$ is the circle determined by $\omega_2$ and $k_n$ is tangent to $\omega,\omega_1,k_{2n-1}$  for $n\ge 1$. Prove that the distance from the center of $k_n$ to $AB$  is $2n$ times the radius of $k_n$.

\item\emph{IMO 1985, \#5.} A circle with center $O$ passes through points $A$ and $C$ and intersects the sides $AB$ and $BC$ of the triangle $ABC$ at points $K$ and $N$, respectively. The circumscribed circles of the triangles $ABC$ and $BNK$ intersect at two distinct points $B$ and $M$. Prove that $\angle BMO=90^{\circ}$.

\item Let $p$ be the semiperimeter of a triangle $ABC$. Points $E$ and $F$ are taken on line $AB$ such that $CE=CF=p$. Prove that the circumcircle of $\triangle CEF$ is tangent to the excircle of $\triangle ABC$ corresponding to $AB$. 

\item Prove that the nine-point circle of a triangle is tangent to the incircle and all three excircles. (Feuerbach s theorem) 

\item The incircle of a triangle $ABC$ is tangent to $BC, CA, AB$ at $M,N$ and $P$, respectively. Show that the circumcenter and incenter of $\triangle ABC$ and the orthocenter of $\triangle MNP$ are collinear. 

\item Points $A,B,C$ are given in this order on a line. Semicircles $k$ and $l$ are drawn on diameters $AB$ and $BC$ respectively, on the same side of the line. A circle $t$ is tangent to $k$ to $l$ at point $T\ne C$, and to the perpendicular $n$ to $AB$ through $C$. Prove that $AT$ is tangent to $l$.

\item\emph{IMO 1997 shortlist.} Let $ A_1A_2A_3$ be a non-isosceles triangle with incenter $ I.$ Let $ C_i,$ $ i = 1, 2, 3,$ be the smaller circle through $ I$ tangent to $ A_iA_{i+1}$ and $ A_iA_{i+2}$ (the addition of indices being mod 3). Let $ B_i, i = 1, 2, 3,$ be the second point of intersection of $ C_{i+1}$ and $ C_{i+2}.$ Prove that the circumcentres of the triangles $ A_1 B_1I,A_2B_2I,A_3B_3I$ are collinear.

\item If seven vertices of a hexahedron lie on a sphere, then so does the eighth vertex. 

\item A sphere with center on the plane of the face $ABC$ of a tetrahedron $SABC$ passes through $A, B$ and $C$, and meets the edges $SA, SB, SC$ again at $A_1, B_1, C_1,$ respectively. The planes through  $A_1, B_1, C_1,$ tangent to the sphere meet at a point $O$. Prove that $O$ is the circumcenter of the tetrahedron $SABC$. 

\item Let $KL$ and $KN$ be the tangents from a point $K$ to a circle $k$. Point $M$ is arbitrarily taken on the extension of $KN$ past $N$, and $P$ is the second intersection point of $k$ with the circumcircle of triangle $KLM$. The point $Q$ is the foot of the perpendicular from $N$ to $ML$. Prove that $\angle MPQ=2\angle KML$.

\item\emph {IMO 2002, G7.} The incircle $ \Omega$ of the acute-angled triangle $ ABC$ is tangent to its side $ BC$ at a point $ K$. Let $ AD$ be an altitude of triangle $ ABC$, and let $ M$ be the midpoint of the segment $ AD$. If $ N$ is the common point of the circle $ \Omega$ and the line $ KM$ (distinct from $ K$), then prove that the incircle $ \Omega$ and the circumcircle of triangle $ BCN$ are tangent to each other at the point $ N$.

\item\emph {IMO 2014, G4.} Consider a fixed circle $\Gamma$ with three fixed points $A, B,$ and $C$ on it. Also, let us fix a real number $\lambda \in(0,1)$. For a variable point $P \not\in\{A, B, C\}$ on $\Gamma$, let $M$ be the point on the segment $CP$ such that $CM =\lambda\cdot CP$ . Let $Q$ be the second point of intersection of the circumcircles of the triangles $AMP$ and $BMC$. Prove that as $P$ varies, the point $Q$ lies on a fixed circle.

\item\emph {IMO 2015,\# 3.} Let $ABC$ be an acute triangle with $AB > AC$. Let $\Gamma $ be its cirumcircle, $H$ its orthocenter, and $F$ the foot of the altitude from $A$. Let $M$ be the midpoint of $BC$. Let $Q$ be the point on $\Gamma$ such that $\angle HQA = 90^{\circ}$ and let $K$ be the point on $\Gamma$ such that $\angle HKQ = 90^{\circ}$. Assume that the points $A$, $B$, $C$, $K$ and $Q$ are all different and lie on $\Gamma$ in this order.

Prove that the circumcircles of triangles $KQH$ and $FKM$ are tangent to each other.

\end{enumerate}

\section {Solutions.}
\begin{enumerate}
\item Let $k_1$ and $k_2$, $k_2$ and $k_3$, $k_3$ and $k_4$, $k_4$ and $k_1$ touch at  $A,B,C,D$ respectively. An inversion with center $A$ maps $k_1$ and $k_2$ to parallel lines $k_1$ and $k_2$, and $k_3$ and $k_4$ to circles $k_3$ and $k_4$ tangent to each other at $C'$ and tangent to $k_2$ at $B'$ and to $k_4$ at $D'$. It is easy to see that $B',C',D'$ are collinear. Therefore $B,C,D$ lie on a circle through $A$. 

\item Applying the inversion with center $A$ and radius $r$ gives  $AB=\dfrac{r^2}{AB'}, CD=\frac{r^2}{AC\prime \cdot AD\prime } C\prime D\prime$, etc. The required inequality reduces to $C\prime D\prime +B\prime C\prime \geq B\prime D\prime$. 

\item Invert through $C$. Semicircle $\omega$ maps to the semicircle $\omega'$ with diameter $P'Q'$, circle $k$ to the tangent to $\omega'$ parallel to $P'Q'$, and line $AB$ to a circle $\ell$ centered on $P'Q'$ which touches $k$ (so it is congruent to the circle determined by $\omega'$). Circle $\ell$ intersects $\omega'$ and $P'Q'$ in $A'$ and $B'$ respectively. Hence $P'A'B'$ is an isosceles triangle with $\angle PAC=\angle A\prime P\prime C= \angle A\prime B\prime C=\angle BAC$. 

\item Under the inversion with center $A$ and squared radius $AB\cdot AC$ points $B$ and $C$ exchange positions, $\omega$ and $\omega_1$ are transformed to the lines perpendicular to $BC$ at $C$ and $B$, and the sequence $(k_n)$ to the sequence of circles $(k_n')$ inscribed in the region between the two lines. Obviously, the distance from the center of $k_n'$ to $AB$ is $2n$ times its radius. Since circle $k_n$ is homothetic to $k_n'$ with respect to $A$, the statement immediately follows. 

\item Invert through $B$. Points $A',C',M'$ are collinear and so are $K\prime ,N\prime ,M\prime$, whereas $A\prime ,C\prime ,N\prime, K\prime$ are on a circle. What does the center $O$ of circle $ACNK$ map to? \emph{Inversion does not preserve centers}. Let $B_1$ and $B_2$ be the feet of the tangents from $B$ to circle $ACNK$. Their images $B_1'$ and $B_2'$ are the feet of the tangents from $B'$ to circle $A'C'N'K'$, and since $O$ lies on the circle $BB_1B_2$, its image $O'$ lies on the line $B_1'B_2'$- more precisely, it is at the midpoint of $B_1'B_2'$. We observe that $M'$ is on the polar of point $B$ with respect to circle $A\prime C\prime N\prime K\prime$, which is nothing but the line $B_1B_2$. It follows that $\angle
OBM=\angle BO\prime M\prime =\angle BO\prime B_1\prime =90^{\circ}$. 

\item The inversion with center $C$ and radius $p$ maps points $E$ and $F$ and the excircle to themselves, and the circumcircle of $\triangle CEF$ to line $AB$ which is tangent to the excircle. The statement follows from the fact that inversion preserves tangency. 

\item We shall show that the nine-point circle $\epsilon$ touches the incircle $k$ and the excircle $k_a$ across $A$. Let $A_1,B_1,C_1$ be the midpoints of $BC,CA,AB$, and $P,Q$ the points of tangency of $k$ and $k_a$ with $BC$, respectively. Recall that $A_1P=A_1Q$; this implies that the inversion with center $A_1$ and radius $A_1P$ takes $k$ and $k_a$ to themselves. This inversion also takes $\epsilon$ to a line. It is not difficult to prove that this line is symmetric to $BC$ with respect to the angle bisector of $\angle BAC$, so it also touches $k$ and $k_a$. 

\item The incenter of $\triangle ABC$ and the orthocenter of $\triangle MNP$ lie on the Euler line of the triangle $ABC$. The inversion with respect to the incircle of $ABC$ maps points $A,B,C$ to the midpoints of $NP,PM,MN$, so the circumcircle of $ABC$ maps to the nine-point circle of $\triangle MNP$ which is also centered on the Euler line of $MNP$. It follows that the center of circle $ABC$ lies on the same line. 

\item An inversion with center $T$ maps circles $t$ and $l$ to parallel lines $t'$ and $l'$, circle $k$ and line $n$ to circles $k'$ and   tangent to $t'$ and $l'$ (where $T\in n'$), and line $AB$ to circle $a'$ perpendicular to $l'$ (because an inversion preserves angles) and passes through $B',C'\in \ell'$; thus $a'$ is the circle with diameter $B'C'$. Circles $k'$ and $n'$ are congruent and tangent to $l'$ at $B'$ and $C'$, and intersect $a'$ at $A'$ and $T$ respectively. It follows that $A'$ and $T$ are symmetric with respect to the perpendicular bisector of $B'C'$ and hence $A'T\parallel \ell'$, so $AT$ is tangent to $\ell$. 

\item The centers of three circles passing through the same point $I$ and not touching each other are collinear if and only if they have another common point. Hence it is enough to show that the circles $A_iB_iI$ have a common point other than $I$. Now apply inversion at center $I$ and with an arbitrary power. We shall denote by $X'$ the image of $X$ under this inversion. In our case, the image of the circle $C_i$ is the line $B_{i+1}'B_{i+2}'$ while the image of the line $A_{i+1}A_{i+2}$  is the circle $IA_{i+1}\prime A_{i+2}\prime$ that is tangent to $B_i\prime B_{i+2}\prime$, and $B_i\prime B_{i+2}\prime$. These three circles have equal radii, so their centers $P_1,P_2,P_3$ form a triangle also homothetic to $\triangle B_1\prime B_2\prime B_3\prime$. Consequently, points $A_1\prime ,A_2\prime ,A_3\prime$, that are the reflections of $I$ across the sides of $P_1P_2P_3$, are vertices of a triangle also homothetic to $B_1\prime B_2\prime B_3\prime$. It follows that $A_1\prime B_1\prime ,A_2\prime B_2\prime ,A_3\prime B_3\prime$ are concurrent at some point $J'$, i.e., that the circles $A_iB_iI$ all pass through $J$. 

\item Let $AYBZ,AZCX,AXDY,WCXD,WDYB,WBZC$  be the faces of the hexahedron, where $A$ is the "eighth" vertex. Apply an inversion with center$W$. Points $B\prime ,C\prime ,D\prime$,$X\prime ,Y\prime ,Z\prime$   lie on some plane $\pi$, and moreover, $C\prime ,X\prime ,D\prime; D\prime ,Y\prime ,B\prime$ ; and $B',Z',C'$ are collinear in these orders. Since $A$ is the intersection of the planes $YBZ,ZCX,XDY$, point $A'$ is the second intersection point of the spheres $WY\prime B\prime Z\prime ,WZ\prime C\prime X\prime ,WX\prime D\prime Y\prime$. Since the circles $Y\prime B\prime Z\prime ,Z\prime C\prime X\prime ,X\prime D\prime Y\prime$,   themselves meet at a point on plane $\pi$, this point must coincide with $A'$. Thus $A'\in\pi$ and the statement follows. 

\item Apply the inversion with center $S$ and squared radius $SA
\cdot SA_1=SB\cdot SB_1=SC\cdot SC_1$. Points $A$ and $A_1$, $B$  and  $B_1$  , and $C$  and $C_1$ map to each other, the sphere through $A,B,C,A_1,B_1,C_1$ maps to itself, and the tangent planes at $A_1,B_1,C_1$  go to the spheres through $S$ and $A$, $S$ and $B$, $S$ and $C$ which touch the sphere $ABCA_1B_1C_1$. These three spheres are perpendicular to the plane $ABC$, so their centers lie on the plane $ABC$; hence they all pass through the point $\overline{S}$  symmetric to $S$ with respect to plane $ABC$. Therefore $\overline{S}$ is the image of $O$. Now since $\angle
SA_1O=\angle S\overline{S}A=\angle\overline{S}SA=\angle OSA_1$, we have $OS=OA_1$ and analogously $OS=OB_1=OC_1$.

\item Apply the inversion with center $M$. Line $MN'$ is tangent to circle $k'$ with center $O'$, and a circle through $M$ is tangent to $k'$ at $L'$ and meets $MN'$ again at $K'$. The line $K'L'$ intersects $k'$ at $P'$, and $N'O'$ intersects $ML'$ at $Q'$. The task is to show that $\angle MQ\prime P\prime = \angle L\prime Q\prime P\prime =2\angle K\prime ML\prime$.

Let the common tangent at $L'$ intersect $MN'$ at $Y'$. Since the peripheral angles on the chords $K'L'$ and $L'P'$ are equal (to $\angle K'L'Y'$), we have $\angle L\prime O\prime P\prime =2\angle L\prime N\prime P\prime =2\angle
K\prime ML\prime$. It only remains to show that $L',P',O',Q'$ are on a circle. This follows from the equality $\angle O\prime Q\prime L\prime =90^{\circ}- \angle
L\prime MK\prime =90^{\circ}-\angle L\prime N\prime P\prime =\angle O\prime P\prime L\prime$ (the angles are regarded as oriented). 

\item Let $k$ be the circle through $B,C$ that is tangent to the circle $\Omega$ at point $N'$. We must prove that $K,M,N'$ are collinear. Since the statement is trivial for $AB=AC$, we may assume that $AC>AB$. As usual, $R,r,\alpha,\beta,\gamma$ denote the circumradius and the inradius and the angles of $\triangle ABC$, respectively.

We have $\tan\angle BKM=DM/DK$. Straightforward calculation gives $DM=\frac12AD=R\sin\beta\sin\gamma$ and $DK$ = $\frac{DC-DB}2$ - $\frac
{KC-KB}2$ = $R\sin(\beta-\gamma)$ - $R(\sin\beta-\sin\gamma)$ = $4R\sin\frac
{\beta-\gamma}2$ $\cdot$ $\sin\frac{\beta}2\sin\frac{\gamma}2$, so we obtain $\tan\angle BKM=\frac{\sin\beta\sin\gamma}
{4\sin\frac{\beta-\gamma}2\sin\frac{\beta}2\sin\frac{\gamma}2}=
\frac{\cos\frac{\beta}2\cos\frac{\gamma}2}{\sin\frac{\beta-\gamma}
2}.$

To calculate the angle $BKN'$, we apply the inversion $\psi$ with center at $K$ and power $BK\cdot CK$. For each object $X$, we denote by $\widehat{X}$ its image under $\psi$. The incircle $\Omega$ maps to a line $\widehat{\Omega}$ parallel to $\widehat{B}\widehat{C}$, at distance $\frac{BK\cdot CK}{2r}$ from $\widehat{B}\widehat{C}$. Thus the point $\widehat{N\prime }$ is the projection of the midpoint $\widehat{U}$ of $\widehat{B}\widehat{C}$ onto $\widehat{\Omega}$. Hence $\tan\angle BKN\prime =\tan
\angle\widehat{B}K\widehat{N\prime }=\frac{\widehat{U}\widehat{N\prime }}
{\widehat{U}K}=\frac{BK\cdot CK}{r(CK-BK)}.$ 

Again, one easily checks that $KB\cdot KC=bc\sin^2\frac{\alpha}2$ and $r=4R\sin\frac{\alpha}2\cdot \sin\frac{\beta}2\cdot
\sin\frac{\gamma}2$, which implies $\tan\angle
BKN\prime =\frac{bc\sin^2\frac{\alpha}2}{r(b-c)}=\frac{4R^2\sin
\beta\sin\gamma\sin^2\frac{\alpha}2}{4R\sin\frac{\alpha}2\sin
\frac{\beta}2\sin\frac{\gamma}2\cdot2R(\sin\beta-\sin\gamma)}=
\frac{\cos\frac{\beta}2\cos\frac{\gamma}2}{\sin\frac{\beta-\gamma}
2}$
                                                
Hence $\angle BKM=\angle BKN\prime$, which implies that $K,M,N'$ are indeed collinear; thus $N'\equiv N$. 

\item Invert about centre $C$ and arbitrary radius, then $\Gamma'$ becomes a line which contains points $A', B', P'$. As $CM=\lambda\cdot CP$, $CP'=\lambda\cdot CM'$. Also, images of circumcircles of $AMP$ and $MBC$ become circumcircle of $A'M'P'$ and line $B'M'$, respectively. Therefore, $Q'$ is the intersection of these two objects.

The power of point theorem yields $B'P'\cdot B'A'=B'M'\cdot B'Q'$, and by our trigonometric trick learnt before we know that\\
$\lambda=\dfrac{CP'}{CM'}$ = $\dfrac{B'P'}{B'M'}\cdot\dfrac{\sin\angle CB'P'}{\sin\angle CB'M'}$ = $\dfrac{B'P'}{B'M'}\cdot\dfrac{\sin\angle CB'A'}{\sin\angle CB'M'}$. Combining the two equalities we have $B'Q'=\lambda\cdot B'A'\cdot\dfrac{\sin\angle CB'M'}{\sin\angle CB'A'}$, whereby all terms except $\angle CB'M'$ are fixed. We therefore write $B'Q'$ as $c\cdot\sin\angle CB'M'$ for some $c\in\mathbb{R}_+$.

We claim that $Q'$ lies on a circle $\omega$ tangent to $CB'$ and with diameter $c$. Let $Q_1$ be such $Q'$ when $\angle CBM'=90^{\circ}$, and clearly $Q_1$ lies on this circle. Now, for any $Q'$ we have $\angle Q'B'Q_1=|\angle CB'Q'-90^{\circ}|$. With $B'Q'=B'Q_1\sin\angle CB'Q'$, we know that triangle $B'Q'Q_1$ has right angle at angle $Q'$. Therefore, this point $Q'$ is on $\omega$, and the locus of $Q'$ is this circle $\omega$.

Finally, since this circle $\omega$ does not pass through $C$ (the only common point of $CB'$ and $\omega$ is the point $B'$), the locus of $Q$ is also a circle.

\item Recall that if $X$ and $Y$ are the foot of perpendicular from $B$ and $C$ to $AC$ and $AB$, respectively, then $AH\cdot HF=BH\cdot HX=CH\cdot HY=k$ for some constant $k$. Now apply negative inversion centred at $H$ and squared radius (-)$k$, which maps $A,B,C$ to $F,X,Y$, respectively. This means that $\Gamma$ is mapped to the nine-point circle of $ABC$. Also notice that $Q,H,M$ are collinear. Indeed, let $QH$ to intersect $\Gamma$ again at $R$, $\angle ABR=\angle ACR=\angle AQR=90^{\circ}$ so $CY, BR\perp AB$, $BX, CR\perp AC$, yielding $BHCR$ a parallelogram. This means $HR$ must bisect $BC$ (i.e. $HR$ passes through $M$!)

Now turn back to the problem. We know that $M$ is on the nine-point circle, and by above $Q$ is mapped to $M$ b this inversion.  Let $L$ be on the nine-point circle with $\angle HML = 90^{\circ}=\angle QKH$, so $K$ is mapped to $L$. The original problem now becomes proving that the line $LM$ is tangent to the circumcircle of $AQL$. Now $LM$ and $AQ$ both perpendicular to $AQ$, so $AQ\parallel LM$. It therefore suffices to prove that $AL=QL$.

 Let $N$, $T$ be midpoints of $HQ$, $AH$. Now $R$ is mapped to $N$ (since $HR\cdot HN=2HM\cdot \frac12 HQ=$ squared radius of negative inversion.) That $T$ is on nine-point circle is already well-known. Moreover, $NT\parallel AQ\parallel LM$, and $NM$ is perpendicular to these three parallel lines. Therefore, $TNML$ is a rectangle. Denote $O$ by centre of $\Gamma$, we know that the nine-point center is the midpoint of $OH$. Since $H$ lies on $NM$, $O$ must lie on $LT$ so $LO\perp AQ.$ With $AO=OQ$ we know that $LT$ is the perpendicular bisector of $AQ$.


\end{enumerate}

\newpage

\section {Appendix: a powerful inversive solution to a very difficult problem.}

\textbf {IMO 2012, G8.} Let $ABC$ be a triangle with circumcircle $\omega$ and $\ell$ a line without common points with $\omega$. Denote by $P$ the foot of the perpendicular from the center of $\omega$ to $\ell$. The side-lines $BC,CA,AB$ intersect $\ell$ at the points $X,Y,Z$ different from $P$. Prove that the circumcircles of the triangles $AXP$, $BYP$ and $CZP$ have a common point different from $P$ or are mutually tangent at $P$.\\
\textbf {Solution.} Invert through $P$ with arbitrary radius. Then the reformulated problem will be the following:\\
\emph {Let $ABC$ be a triangle with circumcircle $\omega$ and $\ell$ a line without common points with $\omega$. Denote by $P$ the foot of the perpendicular from the center of $\omega$ to $\ell$. Denote by $X, Y, Z$ the second intersection of $\ell$ and cirumcircles of $BCP, CAP, ABP$. Prove that $AX, BY, CZ$ are either concurrent or parallel.}

Now denote the intersections of $BC, CA, AB$ and $\ell$ as $D, E, F$, respectively. By the power of point theorem, $DB\cdot DC=DP\cdot DX$. Denote also the radius and centre of $\omega$ as $r$ and $O$, and denote by $d$ the distance $OP$. Let the power of point of any point $G$ to $\omega$ as $f(G)$. Then $DP\cdot (DP+PX) = DP\cdot DX$ = $DB\cdot DC$ = $f(D)=OD^2-r^2$ = $DP^2+OP^2-r^2$. This means $DP\cdot PX=OP^2-r^2=f(P)$, with $P$ lying between $D$ and $X$. Similarly, $EP\cdot PY$ = $FP\cdot PZ$ = $f(P)$.

\usetikzlibrary{arrows}
\pagestyle{empty}
\definecolor{uuuuuu}{rgb}{0.26666666666666666,0.26666666666666666,0.26666666666666666}
\definecolor{qqqqff}{rgb}{0.3333333333333333,0.3333333333333333,0.3333333333333333}
\begin{tikzpicture}[line cap=round,line join=round,>=triangle 45,x=0.4cm,y=0.4cm]
\clip(-2.2466952571383167,-9.204556746808539) rectangle (42.6374977496279,12.535004561686044);
\draw(6.501420496287403,1.7063050049101491) circle (1.2248306811958989cm);
\draw [dotted] (9.899960600250932,1.161456992075047) circle (2.2336779105720725cm);
\draw [dotted] (7.568098165794974,-0.5834834911313422) circle (1.1704491031300404cm);
\draw [dotted] (5.651579620050776,0.5713201504625488) circle (1.6126121771230097cm);
\draw [dash pattern=on 4pt off 4pt] (3.8921135202875,8.643255013186089)-- (4.858830955229784,-3.3815000156604142);
\draw [dash pattern=on 4pt off 4pt] (3.8921135202875,8.643255013186089)-- (8.643519404361546,-3.3048183756139373);
\draw [dash pattern=on 4pt off 4pt] (3.8921135202875,8.643255013186089)-- (13.376010084393346,-3.2089332976935196);
\draw (9.56,1.56)-- (-1.316269766764216,-3.506613841323742);
\draw (5.5,4.6)-- (4.218339408042924,-3.3944770263006063);
\draw (5.5,4.6)-- (15.861773344842037,-3.1585692040196505);
\draw (-1.316269766764216,-3.506613841323742)-- (23.238347206294065,-3.0091123096879975);
\begin{scriptsize}
\draw [fill=qqqqff] (5.5,4.6) circle (2.5pt);
\draw[color=qqqqff] (5.754400017980878,5.3145039475540905) node {$A$};
\draw [fill=qqqqff] (4.645612078977933,-0.7293240418118438) circle (2.5pt);
\draw[color=qqqqff] (4.934775623944278,-0.03256948020849032) node {$B$};
\draw [fill=qqqqff] (9.56,1.56) circle (2.5pt);
\draw[color=qqqqff] (9.85252198816388,2.270184769703862) node {$C$};
\draw [fill=uuuuuu] (6.501420496287403,1.7063050049101491) circle (1.5pt);
\draw[color=uuuuuu] (6.769173077264288,2.270184769703862) node {$O$};
\draw [fill=qqqqff] (6.60378828397404,-3.3461454041573395) circle (2.5pt);
\draw[color=qqqqff] (6.886262276412374,-2.647561594515738) node {$P$};
\draw [fill=uuuuuu] (-1.316269766764216,-3.506613841323742) circle (1.5pt);
\draw[color=uuuuuu] (-1.0367735326080973,-2.9597994589106333) node {$D$};
\draw [fill=uuuuuu] (15.861773344842037,-3.1585692040196505) circle (1.5pt);
\draw[color=uuuuuu] (16.13630900911115,-2.608531861466376) node {$E$};
\draw [fill=uuuuuu] (4.218339408042924,-3.3944770263006063) circle (1.5pt);
\draw[color=uuuuuu] (4.505448560401296,-2.8427102597625478) node {$F$};
\draw [fill=uuuuuu] (13.376010084393346,-3.2089332976935196) circle (1.5pt);
\draw[color=uuuuuu] (13.638406093951989,-2.647561594515738) node {$Z$};
\draw [fill=uuuuuu] (4.858830955229784,-3.3815000156604142) circle (1.5pt);
\draw[color=uuuuuu] (5.129924289191087,-2.8427102597625478) node {$Y$};
\draw [fill=uuuuuu] (8.643519404361546,-3.3048183756139373) circle (1.5pt);
\draw[color=uuuuuu] (8.915808394979194,-2.7646507936638236) node {$X$};
\end{scriptsize}
\end{tikzpicture}

We need to prove that $\dfrac{\sin\angle ABY}{\sin\angle CBY}\cdot\dfrac{\sin\angle BCZ}{\sin\angle ACZ}\cdot\dfrac{\sin\angle CAX}{\sin\angle BAX}=1$. Observe also that $\dfrac{YF}{DY}=\dfrac{BF}{DB}\cdot\dfrac{\sin\angle FBY}{\sin\angle DBY}=\dfrac{\sin\angle FDB}{\sin\angle DFB}\cdot\dfrac{\sin\angle FBY}{\sin\angle DBY}$. Similarly, $\dfrac{XE}{XF}=\dfrac{\sin\angle AFE}{\sin\angle AEF}\cdot\dfrac{\sin\angle EAX}{\sin\angle FAX}$ and $\dfrac{DZ}{EZ}=\dfrac{\sin\angle DEC}{\sin\angle EDC}\cdot \dfrac{\sin\angle DCZ}{\sin\angle ECZ}$. With $\sin\angle FDB=\sin\angle EDC$, $\sin\angle DFB=\sin\angle AFE$, $\sin\angle DEC=\sin\angle AEF$ (each pair of angles is either equal or supplementary) we have $\dfrac{YF}{DY}\cdot\dfrac{XE}{XF}\cdot\dfrac{DZ}{EZ}=\dfrac{\sin\angle FBY}{\sin\angle DBY}\cdot\dfrac{\sin\angle EAX}{\sin\angle FAX}\cdot \dfrac{\sin\angle DCZ}{\sin\angle ECZ}.$ Define a point $G$ in the \emph{extended angle domain} of $\angle ABC$ if $BG$ intersects line $AC$ at a point on \emph{segment} $AC$ (as noted in region $Y$ in the diagram, and $N$ otherwise). Then $\dfrac{\sin\angle FBY}{\sin\angle DBY}:\dfrac{\sin\angle ABY}{\sin\angle CBY}=1$ if segment $DF$ contains all points on extended angle domain of $\angle ABC$, and -1 otherwise.

\usetikzlibrary{arrows}
\definecolor{ffffff}{rgb}{1.,1.,1.}
\definecolor{uuuuuu}{rgb}{0.26666666666666666,0.26666666666666666,0.26666666666666666}
\begin{tikzpicture}[line cap=round,line join=round,>=triangle 45,x=0.05cm,y=0.05cm]
\clip(-34.96341104469802,-37.12157517703881) rectangle (86.45929941783028,21.68925067307286);
\fill[color=uuuuuu,fill=uuuuuu,fill opacity=0.1] (-18.12552278756891,-37.32992792298217) -- (-0.04,4.) -- (21.58518004127728,-36.77323582725069) -- cycle;
\fill[color=uuuuuu,fill=uuuuuu,fill opacity=0.1] (-67.04059774393245,130.32640131391776) -- (54.72275707808309,129.146551166144) -- (-0.04,4.) -- cycle;
\draw [color=uuuuuu] (-18.12552278756891,-37.32992792298217)-- (-0.04,4.);
\draw [color=uuuuuu] (-0.04,4.)-- (21.58518004127728,-36.77323582725069);
\draw [color=ffffff] (21.58518004127728,-36.77323582725069)-- (-18.12552278756891,-37.32992792298217);
\draw [domain=-34.96341104469802:86.45929941783028] plot(\x,{(-73.99528826719494-41.32992792298217*\x)/-18.085522787568912});
\draw [domain=-34.96341104469802:86.45929941783028] plot(\x,{(--84.8697907320191-40.77323582725069*\x)/21.62518004127728});
\draw [color=ffffff] (-67.04059774393245,130.32640131391776)-- (54.72275707808309,129.146551166144);
\draw [color=uuuuuu] (54.72275707808309,129.146551166144)-- (-0.04,4.);
\draw [color=uuuuuu] (-0.04,4.)-- (-67.04059774393245,130.32640131391776);
\draw (-3.287921358821075,-24.134624405829236) node[anchor=north west] {Y};
\draw (-2.9711664619623055,20.950155913735728) node[anchor=north west] {Y};
\draw (-24.61608441397822,3.634221552122955) node[anchor=north west] {N};
\draw (18.99050638691238,4.267731345840495) node[anchor=north west] {N};
\begin{scriptsize}
\draw [fill=uuuuuu] (-0.04,4.) circle (2.5pt);
\draw[color=uuuuuu] (3.363931475213084,4.056561414601315) node {$B$};
\draw [fill=uuuuuu] (-9.423229044208291,-17.44301741437752) circle (2.5pt);
\draw[color=uuuuuu] (-12.3682284021058,-16.74367681245793) node {$A$};
\draw [fill=uuuuuu] (10.099373599662108,-15.117300763763014) circle (2.5pt);
\draw[color=uuuuuu] (12.44423851849781,-14.73756246568572) node {$C$};
\draw[color=ffffff] (-5.7163755680716415,129.17474567381555) node {$b_1$};
\end{scriptsize}
\end{tikzpicture}

We claim that among these three pairs (segment, angle) of $(DF, \angle BAC)$, $(EF,\angle BAC)$ and $(DE, \angle ACB)$, exactly two of the pairs have the segment lying inside the extended angle domain of their corresponding angles. Indeed, $DF$ is on the angle domain of $\angle ABC$ iff $D$ and $F$ are each on extensions of $BC$, $BA$ beyond $B$, or on extensions of $CB$, $AB$ beyond $A$ and $C$ (meaning, $B$ is either both further or both nearer from the points compared to $A$ or $C$). Menelaus' theorem states that $\dfrac{DB}{DC}\cdot\dfrac{EC}{EA}\cdot\dfrac{FA}{FB}=-1$. Taking the modulo of the terms we know that there must be a fraction that is less than 1 and a fraction that is more than 1. Take the example of segment $DF$. If (modulus of) $\dfrac{DB}{DC}, \dfrac{FA}{FB}$ both $>1$, then $D$ is further away from $B$ compared to $C$, but nearer to $B$ compared to $A$, hence cannot be on extended angle domain of $\angle ABC$. Similar conclusion can be reached if the fractions are both $<1$. and opposite result (i.e. on extended angle domain) drawn if exactly one of the ratios aforementioned is $<1$. Thus, knowing that the fractions are $\{>1,>1,<1\}$ or $\{<1.<1.>1\}$ finishes the lemma. This means  $\dfrac{\sin\angle ABY}{\sin\angle CBY}\cdot\dfrac{\sin\angle BCZ}{\sin\angle ACZ}\cdot\dfrac{\sin\angle CAX}{\sin\angle BAX}=1$ iff $\dfrac{YF}{DY}\cdot\dfrac{XE}{XF}\cdot\dfrac{DZ}{EZ}=\dfrac{\sin\angle FBY}{\sin\angle DBY}\cdot\dfrac{\sin\angle EAX}{\sin\angle FAX}\cdot \dfrac{\sin\angle DCZ}{\sin\angle ECZ}=-1.$

Finally, let $\ell$ be the real line and $P$ with coordinate 0, $D,E,F$ with coordinates $a,b,c$ and $X,Y,Z$ has coordinates $x,y,z$, which are equal to $\dfrac{-f(P)}{a},\dfrac{-f(P)}{b},\dfrac{-f(P)}{c}$ by first paragraph. Using signed length, $\dfrac{YF}{DY}=-\dfrac{y-c}{y-a}$ = $-\dfrac{\frac{-f(P)}{b}-c}{\frac{-f(P)}{b}-a}$ = $-\dfrac{f(P)+bc}{f(P)+ab}$. Similarly, $\dfrac{XE}{XF}=-\dfrac{f(P)+ab}{f(P)+ac}$ and $\dfrac{DZ}{EZ}=-\dfrac{f(P)+ac}{f(P)+bc}$. The conclusion follows.




\end{document}